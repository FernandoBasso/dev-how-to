% Created 2024-10-04 Fri 09:32
% Intended LaTeX compiler: pdflatex
\documentclass[11pt]{article}
\usepackage[utf8]{inputenc}
\usepackage[T1]{fontenc}
\usepackage{graphicx}
\usepackage{longtable}
\usepackage{wrapfig}
\usepackage{rotating}
\usepackage[normalem]{ulem}
\usepackage{amsmath}
\usepackage{amssymb}
\usepackage{capt-of}
\usepackage{hyperref}
\date{\today}
\title{sudo edit}
\hypersetup{
 pdfauthor={},
 pdftitle={sudo edit},
 pdfkeywords={},
 pdfsubject={},
 pdfcreator={Emacs 29.4 (Org mode 9.6.15)}, 
 pdflang={English}}
\begin{document}

\maketitle
\tableofcontents


\section{sudo edit with vim and respect env vars}
\label{sec:orgfaf2663}

We can edit a file with \texttt{sudo}, but then vim, nvim, emacs, and other
editors will not read the config files from the user. For example,
this \textbf{will not} cause vim to open FILE with the user vim plugins,
configs, colorschemes, etc.

\begin{verbatim}
$ sudo vim FILE
\end{verbatim}

But we can use the \texttt{-{}-{}preserve-env} option (\texttt{-E}) for short.

\begin{verbatim}
$ sudo -E vim FILE
\end{verbatim}

This should preserve current shell environment so the editor
configuration is respected when running with \texttt{sudo}.

It is also possible to use \texttt{sudoedit} (also part of the \texttt{sudo} package and
man page, at least on Arch Linux), which will use the editor defined
with one of the env vars \texttt{VISUAL}, \texttt{EDITOR}, or \texttt{SUDO\_EDITOR}.

\subsection{References}
\label{sec:org7f5a70c}

\begin{itemize}
\item \texttt{man sudo}
\item \href{https://man.archlinux.org/man/sudo.8}{sudo man page}
\end{itemize}
\end{document}
