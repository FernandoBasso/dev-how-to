% Created 2024-10-07 Mon 07:46
% Intended LaTeX compiler: pdflatex
\documentclass[11pt]{article}
\usepackage[utf8]{inputenc}
\usepackage[T1]{fontenc}
\usepackage{graphicx}
\usepackage{longtable}
\usepackage{wrapfig}
\usepackage{rotating}
\usepackage[normalem]{ulem}
\usepackage{amsmath}
\usepackage{amssymb}
\usepackage{capt-of}
\usepackage{hyperref}
\hypersetup{colorlinks=true,linkcolor=black}
\date{\today}
\title{Percentage | Math}
\begin{document}

\maketitle
\tableofcontents


\section{Introduction}
\label{sec:org5feeed2}

In these examples we will use:

\begin{itemize}
\item \(p\) to represent the percent.
\item \(b\) to represent the base or quantity.
\item \(n\) to represent the percentage.
\end{itemize}

\section{Find percentage of a quantity}
\label{sec:org1a88cb1}

\subsection{Example 1}
\label{sec:org4f24d52}

What is 25\% of 60? We know the percentage \(p\) 25, 60 is the base
\(b\), and we want to find the percentage \(n\).

Then, the formula is:

\[\frac{n}{100} \times {x} = p\]

Substituting and solving:

\begin{itemize}
\item \(\frac{25}{100}\times{60}=p\)
\item \(0.25\times{60}=p\)
\item \(15=p\)
\end{itemize}

Therefore, 25\% of 60 is 15, or 15 is 25\% of 60.

\section{Find the percent or ratio}
\label{sec:orge2c15b9}

\subsection{Example 1}
\label{sec:orgf98ea66}

15 is what percent of 60?

We know that the percentage \(n\) is 15, and that the base or quantity
\(b\) is 60. We want to find percent \(p\).

Then, the formula is:

$\backslash$\frac{n}{b}\texttimes{}\{100\} = p$\backslash$]

Substituting and solving:

\begin{itemize}
\item \(\frac{15}{60}\times{100} = p\)
\item \(0.25\times{100} = p\)
\item \(25 = p\)
\end{itemize}

Therefore, 15 is 25\% of 60.

\section{Find the quantity from a percentage and percent}
\label{sec:org1b77077}

15 is 25\% of what quantity?

We know the percentage \(n\) is 15, and that the percent \(p\)
is 25. We want to find the base (or quantity) \(b\).

The formula is:

\[\frac{n}{p}\times{100} = b\]


Substituting and solving:

\begin{itemize}
\item \(\frac{15}{25}\times{100} = b\)
\item \(0.6\times{100} = b\)
\item \(60 = b\)
\end{itemize}

Therefore, 15 is 25\% of 60.
\end{document}
