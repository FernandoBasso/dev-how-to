% Created 2024-10-05 Sat 09:11
% Intended LaTeX compiler: pdflatex
\documentclass[11pt]{article}
\usepackage[utf8]{inputenc}
\usepackage[T1]{fontenc}
\usepackage{graphicx}
\usepackage{longtable}
\usepackage{wrapfig}
\usepackage{rotating}
\usepackage[normalem]{ulem}
\usepackage{amsmath}
\usepackage{amssymb}
\usepackage{capt-of}
\usepackage{hyperref}
\date{\today}
\title{Unicode and UTF-8}
\hypersetup{
 pdfauthor={},
 pdftitle={Unicode and UTF-8},
 pdfkeywords={},
 pdfsubject={},
 pdfcreator={Emacs 29.4 (Org mode 9.6.15)}, 
 pdflang={English}}
\begin{document}

\maketitle
\tableofcontents


\section{Useful Characters}
\label{sec:org1bcf6dc}

All characters are useful, of course, but here I list some that I find
interesting and/or necessary to use in certain situations.

\begin{center}
\begin{tabular}{lrl}
symbol & hex & name\\[0pt]
\hline
— & 2014 & EM DASH\\[0pt]
␠ & 2420 & SYMBOL FOR SPACE\\[0pt]
· & b7 & MIDDLE DOT\\[0pt]
✓ & 2713 & \\[0pt]
✔ & 2714 & \\[0pt]
✖ & 2716 & \\[0pt]
✗ & 2717 & \\[0pt]
💩 & 1f4a9 & \\[0pt]
‽ & 203d & \\[0pt]
λ & 03bb & \\[0pt]
← & 2190 & \\[0pt]
↑ & 2191 & \\[0pt]
→ & 2192 & \\[0pt]
↓ & 2193 & \\[0pt]
☺ & 263a & \\[0pt]
☻ & 263b & \\[0pt]
♩ & 2669 & QUARTER NOTE\\[0pt]
♪ & 266A & EIGHTH NOTE\\[0pt]
♫ & 266B & BEAMED EIGHTH NOTES\\[0pt]
♭ & 266D & MUSIC FLAT SIGN\\[0pt]
♮ & 266E & MUSIC NATURAL SIGN\\[0pt]
♯ & 266F & MUSIC SHARP SIGN\\[0pt]
« & ab & LEFT-POINTING DOUBLE ANGLE QUOTATION MARK\\[0pt]
» & bb & RIGHT-POINTING DOUBLE ANGLE QUOTATION MARK\\[0pt]
¼ & bc & VULGAR FRACTION ONE QUARTER\\[0pt]
½ & bd & VULGAR FRACTION ONE HALF\\[0pt]
¾ & be & VULGAR FRACTION THREE QUARTERS\\[0pt]
⍽ & 237d & shouldered open box, indicates NBSP.\\[0pt]
· & b7 & interpunct \&middot;\\[0pt]
␠ & 2420 & SYMBOL FOR SPACE\\[0pt]
\end{tabular}
\end{center}


\section{Songs}
\label{sec:org49178cd}

Some pieces of lyrics from some songs that I sometimes would send on a
chat, and wanted to have the verse decorated with some music notation.

Some I would send to my wife from time to time.

\subsection{The Free Software Song by Richard Stallman}
\label{sec:org5eb002d}

\begin{itemize}
\item \href{https://www.gnu.org/music/free-software-song.en.html}{Free Software Song on www.gnu.org}.
\end{itemize}

\begin{quote}
♫ Join us and share the software, ♪
♪ you'll be free hackers\ldots{} ♩ ♬

♫ ♩ Join us now and share the software; You'll be free, hackers, you'll be free. ♬ ♭ ♪
\end{quote}

\subsection{Linda}
\label{sec:org8eef038}

By the group Roupa Nova.

\begin{quote}
♫ ♩ Linda, só você me fascina! ♬ ♪ ♭
\end{quote}

\subsection{Coração Pirata}
\label{sec:org699a5c0}

By the group Roupa Nova.

\begin{quote}
♫ ♩
As pessoas se convencem, de que a sorte me ajudou
Mas plantei cada semente, que o meu coração desejou
♬ ♭ ♪
\end{quote}

\subsection{Te Amo Guria}
\label{sec:org856ee53}

A song by “Grupo Minuano”, from a time where Brazil south music had
not become a pile of crap and noise (like most “popular” music in the
recent decades).

\begin{quote}
♫ ♩ ♬ ♭ ♪
♫ ♩ Eu te amo, te amo demais
Teu amor só me traz alegria
O teu corpo é minha querência
O meu céu minha estrela guia ♬ ♪
\end{quote}
\end{document}
