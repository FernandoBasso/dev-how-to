% Created 2024-10-03 Thu 10:09
% Intended LaTeX compiler: pdflatex
\documentclass[11pt]{article}
\usepackage[utf8]{inputenc}
\usepackage[T1]{fontenc}
\usepackage{graphicx}
\usepackage{longtable}
\usepackage{wrapfig}
\usepackage{rotating}
\usepackage[normalem]{ulem}
\usepackage{amsmath}
\usepackage{amssymb}
\usepackage{capt-of}
\usepackage{hyperref}
\date{\today}
\title{Maybe :: Haskell}
\hypersetup{
 pdfauthor={},
 pdftitle={Maybe :: Haskell},
 pdfkeywords={},
 pdfsubject={},
 pdfcreator={Emacs 29.4 (Org mode 9.6.15)}, 
 pdflang={English}}
\begin{document}

\maketitle
\tableofcontents


\section{Maybe}
\label{sec:orga9dcb7a}

The \texttt{Maybe} data type can either take an argument, or not. Some
languages call this type \texttt{Either} for this reason. “Either we have a
valid/useful value, or we don't.” “Maybe we have a valid/useful value,
but maybe not.” That is the idea behind it.

\section{Pattern Matching on Maybe}
\label{sec:orgd853d7a}

Given \texttt{False}, \texttt{f} returns \texttt{Just 0}, otherwise it returns \texttt{Nothing}. Then, \texttt{g}
displays an example of patter matching on \texttt{Maybe} data constructors.

\begin{verbatim}
f :: Bool -> Maybe Int
f False = Just 0
f _     = Nothing

g :: Maybe Int -> String
g (Just n) = "Value: " ++ (show n)
g Nothing  = "Hmm..."
-- λ> g Nothing
-- "Hmm..."
-- λ> g (Just (1 :: Int))
-- "Value: 1"
-- λ> g $ Just (1 :: Int)
-- "Value: 1"
\end{verbatim}
\end{document}
